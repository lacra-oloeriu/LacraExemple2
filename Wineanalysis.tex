\documentclass[]{article}
\usepackage{lmodern}
\usepackage{amssymb,amsmath}
\usepackage{ifxetex,ifluatex}
\usepackage{fixltx2e} % provides \textsubscript
\ifnum 0\ifxetex 1\fi\ifluatex 1\fi=0 % if pdftex
  \usepackage[T1]{fontenc}
  \usepackage[utf8]{inputenc}
\else % if luatex or xelatex
  \ifxetex
    \usepackage{mathspec}
  \else
    \usepackage{fontspec}
  \fi
  \defaultfontfeatures{Ligatures=TeX,Scale=MatchLowercase}
\fi
% use upquote if available, for straight quotes in verbatim environments
\IfFileExists{upquote.sty}{\usepackage{upquote}}{}
% use microtype if available
\IfFileExists{microtype.sty}{%
\usepackage{microtype}
\UseMicrotypeSet[protrusion]{basicmath} % disable protrusion for tt fonts
}{}
\usepackage[margin=1in]{geometry}
\usepackage{hyperref}
\hypersetup{unicode=true,
            pdfborder={0 0 0},
            breaklinks=true}
\urlstyle{same}  % don't use monospace font for urls
\usepackage{color}
\usepackage{fancyvrb}
\newcommand{\VerbBar}{|}
\newcommand{\VERB}{\Verb[commandchars=\\\{\}]}
\DefineVerbatimEnvironment{Highlighting}{Verbatim}{commandchars=\\\{\}}
% Add ',fontsize=\small' for more characters per line
\usepackage{framed}
\definecolor{shadecolor}{RGB}{248,248,248}
\newenvironment{Shaded}{\begin{snugshade}}{\end{snugshade}}
\newcommand{\KeywordTok}[1]{\textcolor[rgb]{0.13,0.29,0.53}{\textbf{#1}}}
\newcommand{\DataTypeTok}[1]{\textcolor[rgb]{0.13,0.29,0.53}{#1}}
\newcommand{\DecValTok}[1]{\textcolor[rgb]{0.00,0.00,0.81}{#1}}
\newcommand{\BaseNTok}[1]{\textcolor[rgb]{0.00,0.00,0.81}{#1}}
\newcommand{\FloatTok}[1]{\textcolor[rgb]{0.00,0.00,0.81}{#1}}
\newcommand{\ConstantTok}[1]{\textcolor[rgb]{0.00,0.00,0.00}{#1}}
\newcommand{\CharTok}[1]{\textcolor[rgb]{0.31,0.60,0.02}{#1}}
\newcommand{\SpecialCharTok}[1]{\textcolor[rgb]{0.00,0.00,0.00}{#1}}
\newcommand{\StringTok}[1]{\textcolor[rgb]{0.31,0.60,0.02}{#1}}
\newcommand{\VerbatimStringTok}[1]{\textcolor[rgb]{0.31,0.60,0.02}{#1}}
\newcommand{\SpecialStringTok}[1]{\textcolor[rgb]{0.31,0.60,0.02}{#1}}
\newcommand{\ImportTok}[1]{#1}
\newcommand{\CommentTok}[1]{\textcolor[rgb]{0.56,0.35,0.01}{\textit{#1}}}
\newcommand{\DocumentationTok}[1]{\textcolor[rgb]{0.56,0.35,0.01}{\textbf{\textit{#1}}}}
\newcommand{\AnnotationTok}[1]{\textcolor[rgb]{0.56,0.35,0.01}{\textbf{\textit{#1}}}}
\newcommand{\CommentVarTok}[1]{\textcolor[rgb]{0.56,0.35,0.01}{\textbf{\textit{#1}}}}
\newcommand{\OtherTok}[1]{\textcolor[rgb]{0.56,0.35,0.01}{#1}}
\newcommand{\FunctionTok}[1]{\textcolor[rgb]{0.00,0.00,0.00}{#1}}
\newcommand{\VariableTok}[1]{\textcolor[rgb]{0.00,0.00,0.00}{#1}}
\newcommand{\ControlFlowTok}[1]{\textcolor[rgb]{0.13,0.29,0.53}{\textbf{#1}}}
\newcommand{\OperatorTok}[1]{\textcolor[rgb]{0.81,0.36,0.00}{\textbf{#1}}}
\newcommand{\BuiltInTok}[1]{#1}
\newcommand{\ExtensionTok}[1]{#1}
\newcommand{\PreprocessorTok}[1]{\textcolor[rgb]{0.56,0.35,0.01}{\textit{#1}}}
\newcommand{\AttributeTok}[1]{\textcolor[rgb]{0.77,0.63,0.00}{#1}}
\newcommand{\RegionMarkerTok}[1]{#1}
\newcommand{\InformationTok}[1]{\textcolor[rgb]{0.56,0.35,0.01}{\textbf{\textit{#1}}}}
\newcommand{\WarningTok}[1]{\textcolor[rgb]{0.56,0.35,0.01}{\textbf{\textit{#1}}}}
\newcommand{\AlertTok}[1]{\textcolor[rgb]{0.94,0.16,0.16}{#1}}
\newcommand{\ErrorTok}[1]{\textcolor[rgb]{0.64,0.00,0.00}{\textbf{#1}}}
\newcommand{\NormalTok}[1]{#1}
\usepackage{graphicx,grffile}
\makeatletter
\def\maxwidth{\ifdim\Gin@nat@width>\linewidth\linewidth\else\Gin@nat@width\fi}
\def\maxheight{\ifdim\Gin@nat@height>\textheight\textheight\else\Gin@nat@height\fi}
\makeatother
% Scale images if necessary, so that they will not overflow the page
% margins by default, and it is still possible to overwrite the defaults
% using explicit options in \includegraphics[width, height, ...]{}
\setkeys{Gin}{width=\maxwidth,height=\maxheight,keepaspectratio}
\IfFileExists{parskip.sty}{%
\usepackage{parskip}
}{% else
\setlength{\parindent}{0pt}
\setlength{\parskip}{6pt plus 2pt minus 1pt}
}
\setlength{\emergencystretch}{3em}  % prevent overfull lines
\providecommand{\tightlist}{%
  \setlength{\itemsep}{0pt}\setlength{\parskip}{0pt}}
\setcounter{secnumdepth}{0}
% Redefines (sub)paragraphs to behave more like sections
\ifx\paragraph\undefined\else
\let\oldparagraph\paragraph
\renewcommand{\paragraph}[1]{\oldparagraph{#1}\mbox{}}
\fi
\ifx\subparagraph\undefined\else
\let\oldsubparagraph\subparagraph
\renewcommand{\subparagraph}[1]{\oldsubparagraph{#1}\mbox{}}
\fi

%%% Use protect on footnotes to avoid problems with footnotes in titles
\let\rmarkdownfootnote\footnote%
\def\footnote{\protect\rmarkdownfootnote}

%%% Change title format to be more compact
\usepackage{titling}

% Create subtitle command for use in maketitle
\newcommand{\subtitle}[1]{
  \posttitle{
    \begin{center}\large#1\end{center}
    }
}

\setlength{\droptitle}{-2em}

  \title{}
    \pretitle{\vspace{\droptitle}}
  \posttitle{}
    \author{}
    \preauthor{}\postauthor{}
    \date{}
    \predate{}\postdate{}
  

\begin{document}

\begin{Shaded}
\begin{Highlighting}[]
\CommentTok{#install.packages("knitr")}
\CommentTok{#hide code as a default setting globaly}
\NormalTok{knitr}\OperatorTok{::}\NormalTok{opts_chunk}\OperatorTok{$}\KeywordTok{set}\NormalTok{(}\DataTypeTok{echo =} \OtherTok{FALSE}\NormalTok{, }\DataTypeTok{include=}\OtherTok{TRUE}\NormalTok{, }\DataTypeTok{results=}\StringTok{"hide"}\NormalTok{)}
\end{Highlighting}
\end{Shaded}

\section{Explore and Summarize Data by Oloeriu
Lacramioara}\label{explore-and-summarize-data-by-oloeriu-lacramioara}

\begin{verbatim}
## 
## Attaching package: 'dplyr'
\end{verbatim}

\begin{verbatim}
## The following objects are masked from 'package:stats':
## 
##     filter, lag
\end{verbatim}

\begin{verbatim}
## The following objects are masked from 'package:base':
## 
##     intersect, setdiff, setequal, union
\end{verbatim}

\begin{verbatim}
## 
## Attaching package: 'gridExtra'
\end{verbatim}

\begin{verbatim}
## The following object is masked from 'package:dplyr':
## 
##     combine
\end{verbatim}

\begin{verbatim}
## 
## Attaching package: 'GGally'
\end{verbatim}

\begin{verbatim}
## The following object is masked from 'package:dplyr':
## 
##     nasa
\end{verbatim}

\subsection{Load the Data}\label{load-the-data}

\subsection{Univariate Plots Section}\label{univariate-plots-section}

Our dataset consists of 12 variables, with 1599 observations. Quality
variable is discrete and the others are continuous.

\includegraphics{Wineanalysis_files/figure-latex/unnamed-chunk-5-1.pdf}
Red wine quality is normally distributed and concentrated around 5 and
6.

\includegraphics{Wineanalysis_files/figure-latex/unnamed-chunk-6-1.pdf}

The distribution of fixed acidity is right skewed, and concentrated
around 7.9

\includegraphics{Wineanalysis_files/figure-latex/unnamed-chunk-7-1.pdf}
The distribution of volatile acidity seem to be unclear whether it is
bimodal or unimodel, right skewed or normal.

\includegraphics{Wineanalysis_files/figure-latex/unnamed-chunk-8-1.pdf}
The distribution of citric acid is not normal

\includegraphics{Wineanalysis_files/figure-latex/unnamed-chunk-9-1.pdf}

The distribution of residual sugar is right skewed, and concentrated
around 2. There are a few outliers in the plot.

\includegraphics{Wineanalysis_files/figure-latex/unnamed-chunk-10-1.pdf}

The distribution of chlorides is normal, and concentrated around 0.08.
The plot has some outliers.

\includegraphics{Wineanalysis_files/figure-latex/unnamed-chunk-11-1.pdf}

The distribution of free sulfur dioxide is right skewed and concentrated
around 14

\includegraphics{Wineanalysis_files/figure-latex/unnamed-chunk-12-1.pdf}

The distribution of total sulfur dioxide is right skewed and
concentrated around 38. There are a few outliers in the plot.

\includegraphics{Wineanalysis_files/figure-latex/unnamed-chunk-13-1.pdf}

The distribution of density is normal and concentrated around 0.9967

\includegraphics{Wineanalysis_files/figure-latex/unnamed-chunk-14-1.pdf}

The distribution of pH is normal and concentrated around 3.310

\includegraphics{Wineanalysis_files/figure-latex/unnamed-chunk-15-1.pdf}

The distribution of sulphates is right skewed and concentrated around
0.6581. The plot has some outliers.

\includegraphics{Wineanalysis_files/figure-latex/unnamed-chunk-16-1.pdf}

The distribution of alcohol is right skewed and concentrated around
10.20

We divide the data into 2 groups: high quality group contains
observations whose quality is 7 or 8, and low quality group has
observations whose quality is 3 or 4. After examining the difference in
each feature between the two groups, we see that volatile acidity,
density, and citric acid may have some correation with quality. Let's
visualize the data to see the difference.

\includegraphics{Wineanalysis_files/figure-latex/unnamed-chunk-17-1.pdf}

\section{Univariate Analysis}\label{univariate-analysis}

What is the structure of your dataset? There are 1,599 red wines in the
dataset with 11 features on the chemical properties of the wine. (
fixed.acidity, volatile.acidity, citric.acid, residual.sugar, chlorides,
free.sulfur.dioxide, total.sulfur.dioxide, density, pH, sulphates,
alcohol, and quality).

\section{Other observations:}\label{other-observations}

The median quality is 6. Most wines have a pH of 3.4 or higher. About
75\% of wine have quality that is lower than 6. The median percent
alcohol content is 10.20 and the max percent alcohol content is 14.90.

What is/are the main feature(s) of interest in your dataset? The main
features in the data set are pH and quality. I'd like to determine which
features are best for predicting the quality of a wine. I suspect pH and
some combination of the other variables can be used to build a
predictive model to grade the quality of wines.

What other features in the dataset do you think will help support your
investigation into your feature(s) of interest? Volatile acidity, citric
acid, and alcohol likely contribute to the quality of a wine. I think
volatile acidity (the amount of acetic acid in wine) and alcohol (the
percent alcohol content of the wine) probably contribute most to the
quality after researching information on wine quality.

Did you create any new variables from existing variables in the dataset?
I created a new variable called ``quality.level'' which is categorically
divided into ``low'', ``average'', and ``high''. This grouping method
will help us detect the difference among each group more easily.

Of the features you investigated, were there any unusual distributions?
Did you perform any operations on the data to tidy, adjust, or change
the form of the data? If so, why did you do this? Having visualized
acitric acid and volatile acidity data, I observed some unusual
distributions, so I guess this fact may have some correlation with the
quality of red wine. Since the data is clean, I did not perform any
cleaning process or modification of the data.

\section{Bivariate Plots Section}\label{bivariate-plots-section}

\includegraphics{Wineanalysis_files/figure-latex/unnamed-chunk-18-1.pdf}

The graph shows a very clear trend; the lower volatile acidity is, the
higher the quality becomes. The correlation coefficient between quality
and volatile acidity is -0.39. This can be explained by the fact that
volatile acidity at too high of levels can lead to an unpleasant,
vinegar taste.

\includegraphics{Wineanalysis_files/figure-latex/unnamed-chunk-19-1.pdf}

The correlation coefficient is 0.226; the graph shows a weak positive
relationship between quality level and citric acid.

\includegraphics{Wineanalysis_files/figure-latex/unnamed-chunk-20-1.pdf}

With the correlation coefficient of 0.476, the graph shows a positive
relationship between alcohol and quality level. Average quality and low
quality wines have their percent alcohol contents concentrated around 10
whereas high quality wines have their percent alcohol contents
concentrated around 12.

\includegraphics{Wineanalysis_files/figure-latex/unnamed-chunk-21-1.pdf}

A weak negative correlation of -0.2 exists between percent alcohol
content and volatile acidity.

\includegraphics{Wineanalysis_files/figure-latex/unnamed-chunk-22-1.pdf}

The correlation coefficient is 0.04, which indicates that there is
almost no relationship between residual sugar and percent alcohol
content. However, if we actually examine winemaking process, we see that
there is a global trend for wines that are made from ripe to overly ripe
grape fruit. To keep wines from staying too sweet, the fermentation
process has to be left to continue until more of the sugar is consumed,
but as a byproduct, more alcohol is present in the wines.

\includegraphics{Wineanalysis_files/figure-latex/unnamed-chunk-23-1.pdf}

There is a negative correlation between citric acid and volatile
acidity.

\includegraphics{Wineanalysis_files/figure-latex/unnamed-chunk-24-1.pdf}

The correlation coefficient is -0.5, so the relationship is quite clear.
As percent alcohol content increases, the density decreases. The reason
is simple: the density of wine is lower than the density of pure water.

\includegraphics{Wineanalysis_files/figure-latex/unnamed-chunk-25-1.pdf}

This graph shows positive relationship between density and fixed
acidity, positive relationship between fixed acidity and citric acid,
negative relationship between pH and acidity.

\section{Bivariate Analysis}\label{bivariate-analysis}

Talk about some of the relationships you observed in this part of the
investigation. How did the feature(s) of interest vary with other
features in the dataset? I observed a negative relationships between
quality level and volatile acidity, and positive correlation between
quality level and alcohol. I am not suprised at this result, because men
tend to grade stronger wines as high quality, whereas wines with low
percent alcohol are often not graded as such. High volatile acidity is
also perceived to be undesirable because it impacts the taste of wines.
Alcohol and volatile acidity don't have any clear relationship between
each other.

Did you observe any interesting relationships between the other features
(not the main feature(s) of interest)? Yes, I observed positive
relationship between density and fixed acidity, positive relationship
between fixed acidity and citric acid, and negative relationship between
pH and fixed acidity. Other variables either show very weak relationship
or do not show any relationship.

What was the strongest relationship you found? Quality is positively and
strongly correlated with alcohol, and it is also negatively correlated
with volatile acidity. Alcohol and volatile acidity could be used in a
model to predict the quality of wine.

\section{Multivariate Plots Section}\label{multivariate-plots-section}

\includegraphics{Wineanalysis_files/figure-latex/unnamed-chunk-26-1.pdf}

The densities of high quality wines are concentrated between 0.994 and
0.998, and the lower part of volatile acidity (y axis)

\includegraphics{Wineanalysis_files/figure-latex/unnamed-chunk-27-1.pdf}

High quality feature seems to be associated with alcohol ranging from 11
to 13, volatile acidity from 0.2 to 0.5, and citric acid from 0.25 to
0.75

\includegraphics{Wineanalysis_files/figure-latex/unnamed-chunk-28-1.pdf}
\includegraphics{Wineanalysis_files/figure-latex/unnamed-chunk-28-2.pdf}
\includegraphics{Wineanalysis_files/figure-latex/unnamed-chunk-28-3.pdf}
\includegraphics{Wineanalysis_files/figure-latex/unnamed-chunk-28-4.pdf}

The distribution of low and average quality wines seem to be
concentrated at fixed acidity values that are between 6 and 10. pH
increases as fixed acidity decreases, and citric acid increases as fixed
acidity increases.

\includegraphics{Wineanalysis_files/figure-latex/unnamed-chunk-29-1.pdf}

High quality wine density line is distinct from the others, and mostly
distributed between 11 and 12.

\includegraphics{Wineanalysis_files/figure-latex/unnamed-chunk-30-1.pdf}

This chart shows a very clear trend; as volatile acidity decreases, the
quality of wine increases. Wines with volatile acidity exceeding 1 are
almost rated as low quality. The linear model of volatile acidity has an
R-squared of 0.152 which means this feature alone does not explain much
of the variability of red wine quality.

R-squared increases by two times after adding alcohol to the linear
model.

\section{Multivariate Analysis}\label{multivariate-analysis}

Quality has a weak positive relationship with alcohol, and weak negative
relationship with volatile acid. The R squared values are low but
p-values are significant; this result indicates that the regression
models have significant variable but explains little of the variability.
The quality of wine does not solely depends on volatile acidity and
alcohol but also other features. Therefore, it is hard to build a
predictive model that can accurately predict the quality of wine.

Talk about some of the relationships you observed in this part of the
investigation. Were there features that strengthened each other in terms
of looking at your feature(s) of interest? When looking at wine quality
level, we see a positive relationship between fixed acidity and citric
acid

Were there any interesting or surprising interactions between features?
Residual sugar, supposed to play an important part in wine taste,
actually has very little impact on wine quality.

OPTIONAL: Did you create any models with your dataset? Discuss the
strengths and limitations of your model. Yes, I created 3 models. Their
p-values are significant; however, the R squared values are under 0.4,
so they do not provide us with enough explanation about the variability
of the response data around their means.

\section{Final Plots and Summary}\label{final-plots-and-summary}

\section{Plot One}\label{plot-one}

\includegraphics{Wineanalysis_files/figure-latex/unnamed-chunk-32-1.pdf}

\section{Description One}\label{description-one}

The distribution of red wine quality appears to be normal. 82.5\% of
wines are rated 5 and 6 (average quality). Although the rating scale is
between 0 and 10, there exists no wine that is rated 1, 2, 9 or 10.

\section{Plot Two}\label{plot-two}

\includegraphics{Wineanalysis_files/figure-latex/unnamed-chunk-33-1.pdf}

\section{Description Two}\label{description-two}

While citric citric do not have a strong correlation with quality, it is
an important component in the quality of wine. Because citric acid is an
organic acid that contributes to the total acidity of a wine, it is
crucial to have a righ amount of citric acid in wine. Adding citric acid
will give the wine ``freshness'' otherwise not present and will
effectively make a wine more acidic. Wines with citric acid exceeding
0.75 are hardly rated as high quality. 50\% of high quality wines have a
relatively high citric acid that ranges between 0.3 and 0.49, whereas
average and low quality wines have lower amount of citric acid.

\section{Plot Three}\label{plot-three}

\includegraphics{Wineanalysis_files/figure-latex/unnamed-chunk-34-1.pdf}

\section{Description Three}\label{description-three}

We observed the opposite direction to which quality levels are heading.
Wine with high percent alcohol content and low volatile acidity tends to
be rated as high quality wine. Based on the result, we can see that the
volatile acidity in wine and percent alcohol content are two important
components in the quality and taste of red wines.

\section{Reflection}\label{reflection}

The wines data set contains information on 1599 wines across twelve
variables from around 2009. I started by understanding the individual
variables in the data set, and then I explored interesting questions and
leads as I continued to make observations on plots. Eventually, I
explored the quality of wines across many variables and tried creating 3
linear models to predict red wine quality.

There was a trend between the volatile acidity of a wine and its
quality. There was also a trend between the alcohol and its quality. For
the linear model, all wines were included since information on quality,
volatile acidity and alcohol were available for all the wines. The third
linear model with 2 variables were able to account for 31.6\% of the
variance in the dataset.

There are very few wines that are rated as low or high quality. We could
improve the quality of our analysis by collecting more data, and
creating more variables that may contribute to the quality of wine. This
will certainly improve the accuracy of the prediction models. Having
said that, we have successfully identified features that impact the
quality of red wine, visualized their relationships and summarized their
statistics.


\end{document}
